\documentclass[12pt,]{article}
\usepackage{lmodern}
\usepackage{amssymb,amsmath}
\usepackage{ifxetex,ifluatex}
\usepackage{fixltx2e} % provides \textsubscript
\ifnum 0\ifxetex 1\fi\ifluatex 1\fi=0 % if pdftex
  \usepackage[T1]{fontenc}
  \usepackage[utf8]{inputenc}
\else % if luatex or xelatex
  \ifxetex
    \usepackage{mathspec}
  \else
    \usepackage{fontspec}
  \fi
  \defaultfontfeatures{Ligatures=TeX,Scale=MatchLowercase}
\fi
% use upquote if available, for straight quotes in verbatim environments
\IfFileExists{upquote.sty}{\usepackage{upquote}}{}
% use microtype if available
\IfFileExists{microtype.sty}{%
\usepackage{microtype}
\UseMicrotypeSet[protrusion]{basicmath} % disable protrusion for tt fonts
}{}
\usepackage[margin=1.0in]{geometry}
\usepackage{hyperref}
\hypersetup{unicode=true,
            pdfborder={0 0 0},
            breaklinks=true}
\urlstyle{same}  % don't use monospace font for urls
\usepackage{graphicx,grffile}
\makeatletter
\def\maxwidth{\ifdim\Gin@nat@width>\linewidth\linewidth\else\Gin@nat@width\fi}
\def\maxheight{\ifdim\Gin@nat@height>\textheight\textheight\else\Gin@nat@height\fi}
\makeatother
% Scale images if necessary, so that they will not overflow the page
% margins by default, and it is still possible to overwrite the defaults
% using explicit options in \includegraphics[width, height, ...]{}
\setkeys{Gin}{width=\maxwidth,height=\maxheight,keepaspectratio}
\IfFileExists{parskip.sty}{%
\usepackage{parskip}
}{% else
\setlength{\parindent}{0pt}
\setlength{\parskip}{6pt plus 2pt minus 1pt}
}
\setlength{\emergencystretch}{3em}  % prevent overfull lines
\providecommand{\tightlist}{%
  \setlength{\itemsep}{0pt}\setlength{\parskip}{0pt}}
\setcounter{secnumdepth}{0}
% Redefines (sub)paragraphs to behave more like sections
\ifx\paragraph\undefined\else
\let\oldparagraph\paragraph
\renewcommand{\paragraph}[1]{\oldparagraph{#1}\mbox{}}
\fi
\ifx\subparagraph\undefined\else
\let\oldsubparagraph\subparagraph
\renewcommand{\subparagraph}[1]{\oldsubparagraph{#1}\mbox{}}
\fi

%%% Use protect on footnotes to avoid problems with footnotes in titles
\let\rmarkdownfootnote\footnote%
\def\footnote{\protect\rmarkdownfootnote}

%%% Change title format to be more compact
\usepackage{titling}

% Create subtitle command for use in maketitle
\newcommand{\subtitle}[1]{
  \posttitle{
    \begin{center}\large#1\end{center}
    }
}

\setlength{\droptitle}{-2em}
  \title{}
  \pretitle{\vspace{\droptitle}}
  \posttitle{}
  \author{}
  \preauthor{}\postauthor{}
  \date{}
  \predate{}\postdate{}

\usepackage{helvet} % Helvetica font
\renewcommand*\familydefault{\sfdefault} % Use the sans serif version of the font
\usepackage[T1]{fontenc}

\usepackage[none]{hyphenat}

\usepackage{setspace}
\doublespacing
\setlength{\parskip}{1em}

\usepackage{lineno}

\usepackage{pdfpages}

\usepackage{amsmath}

\usepackage{mathtools}

\begin{document}

\section{High Fidelity DNA Polymerase Introduces Bias into 16S rRNA Gene
Sequencing
Results}\label{high-fidelity-dna-polymerase-introduces-bias-into-16s-rrna-gene-sequencing-results}

\begin{center}
\vspace{25mm}

Marc A Sze${^1}$ and Patrick D Schloss${^1}$${^\dagger}$

\vspace{20mm}

$\dagger$ To whom correspondence should be addressed: pschloss@umich.edu

$1$ Department of Microbiology and Immunology, University of Michigan, Ann Arbor, MI




\end{center}

Co-author e-mails:

\begin{itemize}
\tightlist
\item
  \href{mailto:marcsze@med.umich.edu}{\nolinkurl{marcsze@med.umich.edu}}
\end{itemize}

\newpage

\linenumbers

\subsection{Abstract}\label{abstract}

\textbf{Background.} Bias is introduced at different stages of the 16S
rRNA gene sequencing workflow. Commonly studied sources of bias include
preservation and DNA extraction methods. Although cycle number and high
fidelity (HiFi) DNA polymerase are studied less often, they are still
important sources of bias in this workflow. Here, we examine how both
cycle number and HiFi DNA polymerase can change the bacterial community
and introduce bias to the final obtained results.

\textbf{Methods.} We extracted DNA from fecal samples (n=4) using a
PowerMag DNA extraction kit with a 10 minute bead beating step and
amplified at 15, 20, 25, 30, and 35 cycles using Accuprime, Kappa,
Phusion, Platinum, or Q5 HiFi DNA polymerase. Amplification of mock
communities (technical replicates n=4) consisting of previously isolated
whole genomes of 8 different bacteria used the same approach. The
analysis initially examined the number of Operational Taxonomic Units
(OTUs) for both fecal samples and mock communities. It also assessed
HiFi DNA polymerase dependent differences in the Bray-Curtis index,
error rate, sequence error prevalence, and chimera prevalence. Our
analysis also examined chimera prevalence correlation with the number of
OTUs.

\textbf{Results.} When analyzing fecal samples we observed a different
number of OTUs between HiFi DNA polymerases at 35 cycles (P-value
\textless{} 0.0001). Our analysis identified these HiFi dependent
differences in the number of OTUs as early as 20 cycles in the mock
communities (P-value = 0.002). Chimera prevalence varied by HiFi DNA
polymerase and this variation persisted after chimera removal using
VSEARCH. We also observed positive correlations between chimera
prevalence and the number of OTUs which was not affected by chimera
removal with VSEARCH.

\textbf{Conclusions.} HiFi DNA polymerase dependent differences in the
number of OTUs and chimera prevalence makes comparison across studies
difficult. Care should be exercised when choosing both HiFi DNA
polymerase and cycle number to be used in 16S rRNA gene sequencing
studies.

\newpage

\subsection{Introduction}\label{introduction}

Accounting for bias is critical in order to reach an understanding of
bacterial community changes using 16S rRNA gene sequencing results.
Differentiating between bias, reproducibility, and standardization is
important since often times these three can be confused and used
interchangeably with each other. Bias can change the observed results in
a way that is reproducible and standardized. For example, if one group
uses one brand of DNA extraction kit for their 16S rRNA gene sequencing,
their results may be biased versus another group not using the same
brand kit but within their group they can still have reproducible
results (Salter et al., 2014). Our 16S rRNA gene sequencing methods are
biased even when these workflows are standardized to increase
reproducibility. In order to interpret specific studies within the
broader context of the overall field, assessing bias at different parts
of the 16S rRNA gene sequencing workflow is critical.

Many parts of the 16S rRNA gene sequencing workflow contribute bias to
the results and are studied extensively. A typical 16S rRNA gene
sequencing workflow can be divided into preservation, extraction, PCR,
and sequencing steps. Generally, not using a preservation media and
leaving samples at room temperature supports overgrowth of low abundance
members of the fecal bacterial community (Amir et al., 2017). Similarly,
this overgrowth can still occur if the preservation media does not
adequately inhibit growth (Song et al., 2016; Luo et al., 2016). Reports
have also shown that changes in specific community members might occur
due to differing susceptibility to freeze thaw cycles amongst microbes
(Gorzelak et al., 2015). Additionally, reagent contamination can add
community members and the contribution of these contaminant members
grows larger with lower biomass samples (Salter et al., 2014). Overall,
biases due to either preservation or extraction tend to be smaller than
the overall biological signal being measured (Song et al., 2016; Bassis
et al., 2017). However, the contribution of PCR bias to this overall
workflow is not well characterized since these studies use the same PCR
approach while varying preservation or extraction method.

Identifying the biases in the PCR stage of 16S rRNA gene sequencing is
important because a large body of literature shows that there are a
variety of steps during PCR that can change the observed results (Eckert
\& Kunkel, 1991; Burkardt, 2000). Many of these sources of biases are
made worse as cycle number increases (Wang \& Wang, 1996; Haas et al.,
2011; Kebschull \& Zador, 2015). For example, the selective
amplification of AT-rich over GC-rich sequences can exaggerate the
difference between 16S rRNA genes higher in AT versus those higher in GC
(Polz \& Cavanaugh, 1998). Both amplification error and non-specific
amplification (e.g.~incorrect amplicon size products) can also increase
as cycle number increases which can drastically change commonly used
diversity measures (Acinas et al., 2005; Santos et al., 2016).
Additionally, chimeras can form from an aborted extension step followed
by a subsequent priming error and secondary extension and will
artificially increase community diversity (Haas et al., 2011). Although
these differences are not necessarily dependent on primer and DNA
polymerase used, there are also biases that are.

The intrinsic properties to primers and DNA polymerases chosen can also
introduce bias. Primers have variable region dependent binding
affinities for different bacteria and depending on the primer pair do
not detect specific bacteria (e.g.~V1-V3 does not detect
\emph{Haemophilus influenzae} and V3-V5 does not detect
\emph{Propionibacterium acnes}) (Sze et al., 2015 (Table S4); Meisel et
al., 2016). Additionally, there are multiple families of DNA polymerases
that have their own error rate and proof reading capacity (Ishino \&
Ishino, 2014). Interestingly, the influence that these different DNA
polymerases can have on the observed 16S rRNA gene sequencing results
have not been well studied like some of the other previously mentioned
sources of PCR-based bias.

A recent study found clear differences between normal and high fidelity
(HiFi) DNA polymerase and that optimization of the PCR protocol could
reduce error and chimera generation (Gohl et al., 2016). This study also
found that regardless of DNA polymerase, the number of Operational
Taxonomic Units (OTUs) or taxa generated were not easily reduced using
the authors chosen bioinformatic pipeline (Gohl et al., 2016). It is
natural to extend this line of inquiry and ask if biases in the number
of OTUs and chimeras are also dependent on the type of HiFi DNA
polymerase. There is some reason to think that this may be the case
since many of these HiFi DNA polymerases come from different families
(e.g. \emph{Taq} belongs to the family A polymerases) and may
intrinsically have different error rates that cannot be completely
removed with modifications (Ishino \& Ishino, 2014).

Although bias introduced due to differences between DNA polymerase and
HiFi DNA polymerase has been investigated for 16S rRNA gene sequencing,
the bias caused due to differences between specific HiFi DNA polymerases
has not been. This study will specifically address how HiFi DNA
polymerases can bias observed bacterial community results derived from
16S rRNA gene sequencing. We will accomplish this by examining if any of
five different types of HiFi DNA polymerases introduce significant
biases into 16S rRNA gene surveys, if this is a cycle dependent
phenomenon, and whether they can be removed using a standard
bioinformatic pipeline.

\newpage

\subsection{Materials \& Methods}\label{materials-methods}

\textbf{\emph{Human and Mock Samples:}} A single fecal sample was
obtained from 4 individuals who were part of the Enterics Research
Investigational Network (ERIN). The processing and storage of these
samples were previously published (Seekatz et al., 2016). Other than
confirmation that none of these individuals had a \emph{Clostridium
difficle} infection, clinical data and other types of meta data were not
utilized or accessed for this study. All samples were extracted using
the MOBIO\textsuperscript{TM} PowerMag Microbiome RNA/DNA extraction kit
(now Qiagen, MD, USA). The ZymoBIOMICS\textsuperscript{TM} Microbial
Community DNA Standard (Zymo, CA, USA) was used for mock communities and
was made up of \emph{Pseudomonas aeruginosa}, \emph{Escherichia coli},
\emph{Salmonella enterica}, \emph{Lactobacillus fermentum},
\emph{Enterococcus faecalis}, \emph{Staphylococcus aureus},
\emph{Listeria monocytogenes}, and \emph{Bacillus subtilis} at equal
genomic DNA abundance
(\url{http://www.zymoresearch.com/microbiomics/microbial-standards/zymobiomics-microbial-community-standards}).

\textbf{\emph{PCR Protocol:}} The five different HiFi DNA polymerases
that were tested included AccuPrime\textsuperscript{TM} (ThermoFisher,
MA, USA), KAPA HIFI (Roche, IN, USA), Phusion (ThermoFisher, MA, USA),
Platinum (ThermoFisher, MA, USA), and Q5 (New England Biolabs, MA, USA).
The PCR cycle conditions for Platinum and Accuprime followed a
previously published protocol (Kozich et al., 2013)
(\url{https://github.com/SchlossLab/MiSeq_WetLab_SOP/blob/master/MiSeq_WetLab_SOP_v4.md}).
The HiFi DNA polymerase activation time was 2 minutes, unless a
different activation was specified. For Kappa and Q5, a previously
published protocol was used (Gohl et al., 2016). For Phusion, the
company defined conditions were used except for extension time, where
the Accuprime and Platinum settings were used.

The cycle conditions for both fecal and mock samples started at 15 and
increased by 5 up to 35 cycles with amplicons used at each 5-step
increase for sequencing. The PCR of fecal DNA samples consisted of all 4
samples at 15, 20, 25, 30, and 35 cycles for each HiFi DNA polymerase
(total sample n=100). The mock communities had 4 replicates at 15, 20,
25, and 35 cycles and 10 replicates for 30 cycles for all HiFi DNA
polymerases (total samples n=130). No mock community sample had enough
PCR product at 15 cycles for adequate 16S rRNA gene sequencing.

\textbf{\emph{Sequence Processing:}} The mothur software program was
used for all sequence processing steps (Schloss et al., 2009). The
protocol has been previously published (Kozich et al., 2013)
(\url{https://www.mothur.org/wiki/MiSeq_SOP}). Two major differences
from the published protocol were the use of VSEARCH instead of UCHIME
for chimera detection and the use of the OptiClust algorithm instead of
average neighbor for OTU generation at 97\% similarity (Edgar et al.,
2011; Rognes et al., 2016; Westcott \& Schloss, 2017). Sequence error
was determined using the `seq.error' command on mock samples before the
`pre.cluster' command, before chimera removal, and after chimera removal
(Schloss et al., 2009; Cole et al., 2013; Rognes et al., 2016).

\textbf{\emph{Analysis Workflow:}} The total number of OTUs was analyzed
after sub-sampling for both the fecal and mock community samples. For
fecal samples, cycle dependent affects on Bray-Curtis indices were
assessed for cycle group and within individual differences from the
previous cycle (e.g.~20 versus 25, 25 versus 30, etc.). These community
based measures for fecal samples were analyzed at 4 different
sub-sampling sequence depths (1000, 5000, 10000, and 15000) while the
mock community samples were analysed at 3 levels (1000, 5000, 10000).
Based on these observations we analyzed potential reasons for these
differences. Analysis of the mock community of each HiFi DNA polymerase
for general sequence error rate, number of sequences with an error, base
substitution, and numbers of chimeras were assessed before the
`pre.cluster' command, before chimera removal, and after chimera
removal. Additionally, the correlation between the number of chimeras
and the number of OTUs was also assessed before the `pre.cluster'
command, before chimera removal, and after chimera removal.

\textbf{\emph{Statistical Analysis:}} All analysis was done with the R
(v 3.4.4) software package (R Core Team, 2017). Data transformation and
graphing was completed using the tidyverse package (v 1.2.1) and colors
selected using the viridis package (v 0.4.1) (Garnier, 2017; Wickham,
2017). Differences in the total number of OTUs were analyzed using an
ANOVA with a tukey post-hoc test. For the fecal samples the data was
normalized to each individual by cycle number to account for the
biological variation between people. Bray-Curtis distance matrices were
generated using mothur after 100 sub-samplings at 1000, 5000, 10000, and
15000 sequence depth. The distance matrix data was analyzed using
PERMANOVA with the vegan package (v 2.4.5) (Oksanen et al., 2017) and
Kruskal-Wallis tests within R. For both error and chimera analysis,
samples were tested using Kruskal-Wallis with a Dunns post-hoc test.
Where applicable correction for multiple comparison utilized the
Benjamini-Hochberg method (Benjamini \& Hochberg, 1995).

\textbf{\emph{Reproducible Methods:}} The code and analysis can be found
here \url{https://github.com/SchlossLab/Sze_PCRSeqEffects_XXXX_2017}.
The raw sequences can be found in the SRA at the following accession
number SRP132931.

\newpage

\subsection{Results}\label{results}

\textbf{\emph{The Number of OTUs are Dependent on HiFi DNA Polymerase:}}
A consistent difference in the number of OTUs, that was dependent on the
HiFi DNA polymerase used was observed regardless of sub-sampling depth
for fecal samples {[}Figure 1{]}. Additionally, there was a trend for
lower cycle numbers (15-20) to result in less differences in the number
of OTUs versus higher cycle numbers (25, 30, and 35) between HiFi DNA
polymerases {[}Figure 1{]}. For fecal samples, all sub-sampling levels
had significant differences between HiFi DNA polymerases at 35 cycles
(F-stat \textgreater{} 16.35, P-value \textless{} 0.0001) {[}Table
S1{]}. Most of the differences observed at 35 cycles were between
Platinum and other HiFi DNA polymerases, based on a Tukey post-hoc test
(P-value \textless{} 0.05) {[}Table S2{]}. Differences in the number of
OTUs between HiFi DNA polymerases were identifiable at earlier cycles
(25 and 30) but the sub-sampling depth had to be 5000 sequences or
higher (F-stat \textgreater{} 4.98, P-value \textless{} 0.05) {[}Table
S1{]}.

This HiFi DNA polymerase dependent difference in the number of OTUs was
also observed in the mock community samples {[}Figure 2{]}. Regardless
if fecal or mock communities were used, the same HiFi DNA polymerases
had high (Platinum) and low (Accuprime) number of OTUs and this was
consistent across cycle number and sub-sampling depth {[}Figure 1-2 \&
Table S1-S4{]}. In contrast to the results obtained with fecal samples,
differences between HiFi DNA polymerases were observed as early as 20
cycles and at as low of a sub-sampling depth as 1000 sequences in the
mock community samples (F-stat = 15.82, P-value = 0.002) {[}Table S3{]}.
For both cycle numbers and sub-sampling depths, the majority of
differences in the number of OTUs were between Platinum and the other
HiFi DNA polymerases {[}Table S4{]}. Based on these observations in
fecal and mock communities, it is clear that different HiFi DNA
polymerases result in a different total number of OTUs observed within a
sample.

\textbf{\emph{Minimal Bray-Curtis Differences are Detected and are
Dependent on both Cycle Number and Sub-Sampling Depth:}} A few small
differences based on sub-sampling and cycle number were detected in
overall bacterial community composition. Within the same fecal sample
and independent of HiFi DNA polymerases, there were differences in the
community composition between 20 and 25 cycles that was dependent on
sub-sampling depth (sub-sampled to 1000 = 0.51 (0.4 - 0.79) (median
(IQR)), sub-sampled to 5000 = 0.43 (0.33 - 0.63), sub-sampled to 10000 =
0.4 (0.24 - 0.43)) {[}Figure 3A{]}. Further, when data was available for
the mock communities, there were larger observed differences between 20
and 25 cycles (sub-sampled to 1000 = 0.88 (0.42 - 0.91)) {[}Figure
3B{]}. Additionally, these stated community differences disappear when
comparing 25 to 30 cycles and do not persist past 25 cycles {[}Figure
3{]}. Although these trends are clearly noticeable, we found that there
was no detectable difference in Bray-Curtis index when comparing to the
previous 5-cycle increment for both fecal and mock communities after
multiple comparison correction (P-value \textgreater{} 0.05). Using
PERMANOVA to test for community differences based on cycle number within
HiFi DNA polymerases, only Phusion had cycle dependent differences at
1000 and 5000 sub-sampling depths (P-value = 0.03 and 0.01). For fecal
samples, Phusion was one of two HiFi DNA polymerases that had enough
sequences to reach a sub-sampling depth of 1000 at 15 cycles. Overall,
these data suggest that there are small HiFi DNA polymerase differences
in Bray-Curtis index that are dependent on both sub-sampling depth and
cycle number.

\textbf{\emph{Sequence Error is Dependent on both HiFi DNA Polymerase
and Cycle Number:}} Differences in the median average per base error
varied by HiFi DNA polymerase without a clear pattern across
sub-sampling depth {[}Table S5{]}. The highest median average per base
error rates were for the Kappa HiFi DNA polymerase {[}Figure 4{]}. This
error rate was minimally affected by both the `pre.cluster' step and
chimera removal by VSEARCH {[}Figure 4{]}. The differences in the median
average per base error rate between the different HiFi DNA polymerases
was cycle dependent with Platinum having the largest changes versus
other HiFi DNA polymerases {[}Figure 4B-C and Table S6{]}. The total
sequences with at least one error was also cycle number dependent and
differences between HiFi DNA polymerases could be drastically reduced by
the use of the `pre.cluster' step {[}Figure S1{]}. These differences in
sequences with at least one error were mostly due to differences in
Accuprime\textsuperscript{TM} and Platinum versus the other HiFi DNA
polymerases {[}Figure S1 \& Table S7 \& S8{]}. Finally, we did not
observe a HiFi DNA polymerase dependent difference on base substitution
rate {[}Figure S2{]}. Although sequence error is dependent on HiFi DNA
polymerase some of these error dependent differences can be corrected
using existing bioinformatic approaches.

\textbf{\emph{Prevalence of Chimeric Sequences are HiFi DNA Polymerase
Dependent and Correlate with the Number of OTUs:}} There were
significant differences in the chimera prevalence based on HiFi DNA
polymerase used at all levels of sub-sampling and cycle numbers (P-value
\textless{} 0.05) {[}Table S9{]}. Differences in chimera prevalence
between Platinum and all other HiFi DNA polymerases accounted for the
majority of these differences {[}Table S10{]}.
Accuprime\textsuperscript{TM} had the lowest chimera prevalence of all
HiFi DNA polymerases regardless of whether `pre.cluster' or chimera
removal with VSEARCH was used {[}Figure 5{]}. A positive correlation was
observed between chimeric sequences and the number of OTUs for all HiFi
DNA polymerases {[}Figure 6{]}. This positive correlation was strongest
for Accuprime\textsuperscript{TM}, Platinum, and Phusion HiFi DNA
Polymerases {[}Figure 6{]}. The R\textsuperscript{2} value between the
number of OTUs and chimeric sequences did not change with the use of
`pre.cluster' or with the removal of chimeras using VSEARCH {[}Figure
6{]}. This data suggests that chimera prevalence depends on HiFi DNA
polymerase used and confirms that the number of OTUs is dependent on the
prevalence of these chimeric sequences.

\newpage

\subsection{Discussion}\label{discussion}

In this study we show that the number of OTUs, error rate, and chimera
prevalence depends on HiFi DNA polymerase {[}Figure 1-2 \& 4-5{]}. These
differences are important because many diversity metrics rely on the
number of OTUs or other measures dependent on error rate and chimera
prevalence as part of their metric calculations (e.g.~richness).
Additionally, the earlier detection of differences in total number of
OTUs between HiFi DNA polymerases in the mock versus fecal samples might
indicate that high biomass samples may underestimate the biases present
within low biomass samples. We observed that undetected chimeras that
were not identified and removed using standard bioinformatic approaches
cause many of these differences. This suggests that specific diversity
differences between studies can be attributed to differences in HiFi DNA
polyermase used. Based on our observations metrics that measure within
sample diversity depend on HiFi DNA polymerase but this may not be the
case for metrics that assess between sample diversity.

There were few differences that depend on HiFi DNA polymerase for
between sample diversity, as measured by the Bray-Curtis index. Using
this metric our observations found no differences in the overall
bacterial community composition for sub-sampling depth or cycle number
used. One possible reason for this outcome was that our study did not
have enough power to detect differences due to low sample number in each
group. Another reason was that many of the OTUs are likely not highly
abundant, allowing the Bray-Curtis index to be able to successfully
down-weight chimeric OTUs (Minchin, 1987). The choice of downstream
diversity metric could be an important consideration in helping to
mitigate these observed HiFi DNA polymerase dependent differences in
chimera prevalence. Metrics that solely use presence/absence of OTUs
(e.g.~Jaccard (Real \& Vargas, 1996)) may be less robust to chimera
prevalence and by extension total number of OTU differences in HiFi DNA
Polymerases. When choosing a distance metric, careful consideration of
the biases introduced from the PCR step of the 16S rRNA gene sequencing
workflow need to be taken into account. With differences in the number
of OTUs and chimera prevalence depending on HiFi DNA polymerase used, it
might be easier to avoid specific DNA polymerase families altogether.

Although the variation in error rate and chimera prevalence may be due
to the DNA polymerase family, the highest and lowest chimera rates both
belonged to a family A polymerase (Platinum and
Accuprime\textsuperscript{TM} respectively) (Ishino \& Ishino, 2014).
Additionally, based on the material safety data sheet (MSDS) the
differences between the two HiFi DNA polymerases are not immediately
apparent. Both HiFi DNA polymerases contain a recombinant \emph{Taq} DNA
polymerase, a \emph{Pyrococcus} spp GB-D polymerase and a platinum
\emph{Taq} antibody. With everything else being equal, it is possible
that differences in how the recombinant \emph{Taq} was generated could
be a contributing factor for the differences observed between the HiFi
DNA polymerases. We are unlikely to avoid adding HiFi dependent bias to
16S rRNA gene sequencing results, however, these differences may not be
large enough to mask the actual biological signal.

The majority of HiFi DNA polymerases we studied add small increases in
the number of OTUs and chimera prevalence that can be masked by
biological differences. The sequence error introduced by the HiFi DNA
polymerase is small and likely to be smaller than the biological
variation within a specific study, which would be consistent with
previous findings for preservation and DNA extraction methods (Salter et
al., 2014; Song et al., 2016; Luo et al., 2016). However, the chimarea
prevalence for some HiFi DNA polymerases (e.g.~Platinum) are relatively
large and might be greater than the oberseved biological variation
within a specific study. The choice of HiFi DNA polymerase can be as
important a consideration as either preservation or DNA extraction
method used because similar to using different preservation methods or
different DNA extraction kits, the type of HiFi DNA polymerase can add
bias to the observed bacterial community. Although avoiding HiFi DNA
polymerases that yield a high number of OTUs and chimera prevalence
might provide better standardization of results it does come at a cost.

Methods can be standardized but they commonly contain bias that is
reproducible and may miss important associations. Bias can be easily
reproduced and can be found in every step of the 16S rRNA gene
sequencing workflow (Salter et al., 2014; Gohl et al., 2016; Luo et al.,
2016; Amir et al., 2017). This study shows that specific diversity
metrics used to measure the microbial community consistently vary based
on HiFi DNA polymerase. Standardizing multiple workflows to one specific
HiFi DNA polymerase could be detrimental since some HiFi DNA polymerases
may work better in certain situations over others. Arguably, the degree
of workflow standardization across studies and research group needs to
be approached on a study by study basis and not every project needs to
use the exact same approach. All aspects of the 16S rRNA gene sequencing
workflow need to be customized for the specific microbial community that
will be sampled. Although a diversity of approaches may make
reproducibility and replicability more difficult it will help to avoid
systematic biases from occuring due to widespread standardization of
approaches.

\newpage

\subsection{Conclusion}\label{conclusion}

Our observations fill a gap in knowledge on the bias introduced to 16S
rRNA gene sequencing results due to differences in HiFi DNA polymerases.
We found that the number of OTUs and the chimera prevalence is dependent
on both HiFi DNA polymerase and cycle number chosen. Care should be
taken when choosing a HiFi DNA polymerase for 16S rRNA gene surveys
because their intrinsic differences can change the number of OTUs
observed and influence diversity based metrics that do not down-weight
rare observations. Knowing the inherent bias associated with different
HiFi DNA polymerases allows for better interpretation of the relationsip
between an individual study to their respective field of research.

\newpage

\subsection{Acknowledgements}\label{acknowledgements}

The authors would like to thank all the study participants in ERIN whose
samples were utilized. We would also like to thank Judy Opp and April
Cockburn for their effort in sequencing the samples as part of the
Microbiome Core Facility at the University of Michigan. Salary support
for Marc A. Sze came from the Canadian Institute of Health Research and
NIH grant UL1TR002240. Salary support for Patrick D. Schloss came from
NIH grants P30DK034933 and 1R01CA215574.

\newpage

\subsection{References}\label{references}

\hypertarget{refs}{}
\hypertarget{ref-Acinas2005}{}
Acinas SG., Sarma-Rupavtarm R., Klepac-Ceraj V., Polz MF. 2005.
PCR-induced sequence artifacts and bias: Insights from comparison of two
16S rRNA clone libraries constructed from the same sample. \emph{Applied
and Environmental Microbiology} 71:8966--8969. DOI:
\href{https://doi.org/10.1128/aem.71.12.8966-8969.2005}{10.1128/aem.71.12.8966-8969.2005}.

\hypertarget{ref-Amir2017}{}
Amir A., McDonald D., Navas-Molina JA., Debelius J., Morton JT., Hyde
E., Robbins-Pianka A., Knight R. 2017. Correcting for microbial blooms
in fecal samples during room-temperature shipping. \emph{mSystems}
2:e00199--16. DOI:
\href{https://doi.org/10.1128/msystems.00199-16}{10.1128/msystems.00199-16}.

\hypertarget{ref-storage_Bassis_2017}{}
Bassis CM., Nicholas M. Moore., Lolans K., Seekatz AM., Weinstein RA.,
Young VB., Hayden MK. 2017. Comparison of stool versus rectal swab
samples and storage conditions on bacterial community profiles.
\emph{BMC Microbiology} 17. DOI:
\href{https://doi.org/10.1186/s12866-017-0983-9}{10.1186/s12866-017-0983-9}.

\hypertarget{ref-benjamini_controlling_1995}{}
Benjamini Y., Hochberg Y. 1995. Controlling the false discovery rate: A
practical and powerful approach to multiple testing. \emph{Journal of
the Royal Statistical Society. Series B (Methodological)} 57:289--300.

\hypertarget{ref-Burkardt2000}{}
Burkardt H-J. 2000. Standardization and quality control of PCR analyses.
\emph{Clinical Chemistry and Laboratory Medicine} 38. DOI:
\href{https://doi.org/10.1515/cclm.2000.014}{10.1515/cclm.2000.014}.

\hypertarget{ref-rdp_Cole_2013}{}
Cole JR., Wang Q., Fish JA., Chai B., McGarrell DM., Sun Y., Brown CT.,
Porras-Alfaro A., Kuske CR., Tiedje JM. 2013. Ribosomal database
project: Data and tools for high throughput rRNA analysis. \emph{Nucleic
Acids Research} 42:D633--D642. DOI:
\href{https://doi.org/10.1093/nar/gkt1244}{10.1093/nar/gkt1244}.

\hypertarget{ref-Eckert1991}{}
Eckert KA., Kunkel TA. 1991. DNA polymerase fidelity and the polymerase
chain reaction. \emph{Genome Research} 1:17--24. DOI:
\href{https://doi.org/10.1101/gr.1.1.17}{10.1101/gr.1.1.17}.

\hypertarget{ref-uchime_Edgar_2011}{}
Edgar RC., Haas BJ., Clemente JC., Quince C., Knight R. 2011. UCHIME
improves sensitivity and speed of chimera detection.
\emph{Bioinformatics} 27:2194--2200. DOI:
\href{https://doi.org/10.1093/bioinformatics/btr381}{10.1093/bioinformatics/btr381}.

\hypertarget{ref-viridis_citation_2017}{}
Garnier S. 2017. \emph{Viridis: Default color maps from 'matplotlib'}.

\hypertarget{ref-taq_Gohl_2016}{}
Gohl DM., Vangay P., Garbe J., MacLean A., Hauge A., Becker A., Gould
TJ., Clayton JB., Johnson TJ., Hunter R., Knights D., Beckman KB. 2016.
Systematic improvement of amplicon marker gene methods for increased
accuracy in microbiome studies. \emph{Nature Biotechnology} 34:942--949.
DOI: \href{https://doi.org/10.1038/nbt.3601}{10.1038/nbt.3601}.

\hypertarget{ref-Gorzelak2015}{}
Gorzelak MA., Gill SK., Tasnim N., Ahmadi-Vand Z., Jay M., Gibson DL.
2015. Methods for improving human gut microbiome data by reducing
variability through sample processing and storage of stool. \emph{PLOS
ONE} 10:e0134802. DOI:
\href{https://doi.org/10.1371/journal.pone.0134802}{10.1371/journal.pone.0134802}.

\hypertarget{ref-Haas2011}{}
Haas BJ., Gevers D., Earl AM., Feldgarden M., Ward DV., Giannoukos G.,
Ciulla D., Tabbaa D., Highlander SK., Sodergren E., Methe B., DeSantis
TZ., Petrosino JF., Knight R., and BWB. 2011. Chimeric 16S rRNA sequence
formation and detection in sanger and 454-pyrosequenced PCR amplicons.
\emph{Genome Research} 21:494--504. DOI:
\href{https://doi.org/10.1101/gr.112730.110}{10.1101/gr.112730.110}.

\hypertarget{ref-polymerase_Ishino_2014}{}
Ishino S., Ishino Y. 2014. DNA polymerases as useful reagents for
biotechnology â the history of developmental research in the field.
\emph{Frontiers in Microbiology} 5. DOI:
\href{https://doi.org/10.3389/fmicb.2014.00465}{10.3389/fmicb.2014.00465}.

\hypertarget{ref-Kebschull2015}{}
Kebschull JM., Zador AM. 2015. Sources of PCR-induced distortions in
high-throughput sequencing data sets. \emph{Nucleic Acids
Research}:gkv717. DOI:
\href{https://doi.org/10.1093/nar/gkv717}{10.1093/nar/gkv717}.

\hypertarget{ref-protocol_Kozich_2013}{}
Kozich JJ., Westcott SL., Baxter NT., Highlander SK., Schloss PD. 2013.
Development of a dual-index sequencing strategy and curation pipeline
for analyzing amplicon sequence data on the MiSeq illumina sequencing
platform. \emph{Applied and Environmental Microbiology} 79:5112--5120.
DOI: \href{https://doi.org/10.1128/aem.01043-13}{10.1128/aem.01043-13}.

\hypertarget{ref-preservation_Luo_2016}{}
Luo T., Srinivasan U., Ramadugu K., Shedden KA., Neiswanger K., Trumble
E., Li JJ., McNeil DW., Crout RJ., Weyant RJ., Marazita ML., Foxman B.
2016. Effects of specimen collection methodologies and storage
conditions on the short-term stability of oral microbiome taxonomy.
\emph{Applied and Environmental Microbiology} 82:5519--5529. DOI:
\href{https://doi.org/10.1128/aem.01132-16}{10.1128/aem.01132-16}.

\hypertarget{ref-Meisel2016}{}
Meisel JS., Hannigan GD., Tyldsley AS., SanMiguel AJ., Hodkinson BP.,
Zheng Q., Grice EA. 2016. Skin microbiome surveys are strongly
influenced by experimental design. \emph{Journal of Investigative
Dermatology} 136:947--956. DOI:
\href{https://doi.org/10.1016/j.jid.2016.01.016}{10.1016/j.jid.2016.01.016}.

\hypertarget{ref-bc_index_Minchin1987}{}
Minchin PR. 1987. An evaluation of the relative robustness of techniques
for ecological ordination. \emph{Vegetatio} 69:89--107. DOI:
\href{https://doi.org/10.1007/bf00038690}{10.1007/bf00038690}.

\hypertarget{ref-vegan_citation}{}
Oksanen J., Blanchet FG., Friendly M., Kindt R., Legendre P., McGlinn
D., Minchin PR., O'Hara RB., Simpson GL., Solymos P., Stevens MHH.,
Szoecs E., Wagner H. 2017. \emph{Vegan: Community ecology package}.

\hypertarget{ref-polz_bias_1998}{}
Polz MF., Cavanaugh CM. 1998. Bias in template-to-product ratios in
multitemplate PCR. \emph{Applied and Environmental Microbiology}
64:3724--3730.

\hypertarget{ref-r_citation_2017}{}
R Core Team. 2017. \emph{R: A language and environment for statistical
computing}. Vienna, Austria: R Foundation for Statistical Computing.

\hypertarget{ref-Real1996}{}
Real R., Vargas JM. 1996. The probabilistic basis of jaccards index of
similarity. \emph{Systematic Biology} 45:380--385. DOI:
\href{https://doi.org/10.1093/sysbio/45.3.380}{10.1093/sysbio/45.3.380}.

\hypertarget{ref-vsearch_Rognes_2016}{}
Rognes T., Flouri T., Nichols B., Quince C., Mahé F. 2016. VSEARCH: A
versatile open source tool for metagenomics. \emph{PeerJ} 4:e2584. DOI:
\href{https://doi.org/10.7717/peerj.2584}{10.7717/peerj.2584}.

\hypertarget{ref-contamination_Salter2014}{}
Salter SJ., Cox MJ., Turek EM., Calus ST., Cookson WO., Moffatt MF.,
Turner P., Parkhill J., Loman NJ., Walker AW. 2014. Reagent and
laboratory contamination can critically impact sequence-based microbiome
analyses. \emph{BMC Biology} 12. DOI:
\href{https://doi.org/10.1186/s12915-014-0087-z}{10.1186/s12915-014-0087-z}.

\hypertarget{ref-BautistadelosSantos2016}{}
Santos QMB-d los., Schroeder JL., Blakemore O., Moses J., Haffey M.,
Sloan W., Pinto AJ. 2016. The impact of sampling, PCR, and sequencing
replication on discerning changes in drinking water bacterial community
over diurnal time-scales. \emph{Water Research} 90:216--224. DOI:
\href{https://doi.org/10.1016/j.watres.2015.12.010}{10.1016/j.watres.2015.12.010}.

\hypertarget{ref-mothur_schloss_2009}{}
Schloss PD., Westcott SL., Ryabin T., Hall JR., Hartmann M., Hollister
EB., Lesniewski RA., Oakley BB., Parks DH., Robinson CJ., Sahl JW.,
Stres B., Thallinger GG., Horn DJV., Weber CF. 2009. Introducing mothur:
Open-source, platform-independent, community-supported software for
describing and comparing microbial communities. \emph{Applied and
Environmental Microbiology} 75:7537--7541. DOI:
\href{https://doi.org/10.1128/aem.01541-09}{10.1128/aem.01541-09}.

\hypertarget{ref-erin_seekatz_2016}{}
Seekatz AM., Rao K., Santhosh K., Young VB. 2016. Dynamics of the fecal
microbiome in patients with recurrent and nonrecurrent clostridium
difficile infection. \emph{Genome Medicine} 8. DOI:
\href{https://doi.org/10.1186/s13073-016-0298-8}{10.1186/s13073-016-0298-8}.

\hypertarget{ref-preservation_Song_2016}{}
Song SJ., Amir A., Metcalf JL., Amato KR., Xu ZZ., Humphrey G., Knight
R. 2016. Preservation methods differ in fecal microbiome stability,
affecting suitability for field studies. \emph{mSystems} 1:e00021--16.
DOI:
\href{https://doi.org/10.1128/msystems.00021-16}{10.1128/msystems.00021-16}.

\hypertarget{ref-Sze2015}{}
Sze MA., Dimitriu PA., Suzuki M., McDonough JE., Campbell JD., Brothers
JF., Erb-Downward JR., Huffnagle GB., Hayashi S., Elliott WM., Cooper
J., Sin DD., Lenburg ME., Spira A., Mohn WW., Hogg JC. 2015. Host
response to the lung microbiome in chronic obstructive pulmonary
disease. \emph{American Journal of Respiratory and Critical Care
Medicine} 192:438--445. DOI:
\href{https://doi.org/10.1164/rccm.201502-0223oc}{10.1164/rccm.201502-0223oc}.

\hypertarget{ref-Wang1996}{}
Wang GCY., Wang Y. 1996. The frequency of chimeric molecules as a
consequence of PCR co-amplification of 16S rRNA genes from different
bacterial species. \emph{Microbiology} 142:1107--1114. DOI:
\href{https://doi.org/10.1099/13500872-142-5-1107}{10.1099/13500872-142-5-1107}.

\hypertarget{ref-opticlust_Westcott_2017}{}
Westcott SL., Schloss PD. 2017. OptiClust, an improved method for
assigning amplicon-based sequence data to operational taxonomic units.
\emph{mSphere} 2:e00073--17. DOI:
\href{https://doi.org/10.1128/mspheredirect.00073-17}{10.1128/mspheredirect.00073-17}.

\hypertarget{ref-tidyverse_2017}{}
Wickham H. 2017. \emph{Tidyverse: Easily install and load 'tidyverse'
packages}.

\newpage

\textbf{Figure 1: Normalized Fecal Number of OTUs.} The x-axis
represents the different sub-sampling depths used and the y-axis is the
normalized within individual number of OTUs. The red line represents the
overall mean Z-score normalized number of OTUs for each respective HiFi
DNA polymerase. The dashed black line represents the overall Z-score
normalized mean number of OTUs.

\textbf{Figure 2: Mock Sample Variability in Number of OTUs based on
HiFi DNA Polymerase.} A) Sub-sampled to 1000 reads. B) Sub-sampled to
5000 reads. C) Sub-sampled to 10000 reads. The dotted line represents
the number of OTUs generated when the mock reference sequences are run
through the pipeline.

\textbf{Figure 3: Community Differences by Five-Cycle Intervals and
Sub-sampling Depth.} A) Fecal samples within person difference based on
the next 5-cycle PCR interval. B) Mock samples within replicate
difference based on the next 5-cycle PCR interval.

\textbf{Figure 4: HiFi DNA Polymerase Per Base Error Rate in Mock
Samples.} A) Error rate before the merger of sequences with pre.cluster
and the removal of chimeras with VSEARCH. B) Error rate before the
removal of chimeras with VSEARCH. C) Full pipeline. The error bars
represent the 75\% interquartile range of the median.

\textbf{Figure 5: HiFi DNA Polymerase Chimera Prevalence in Mock
Samples.} A) Chimera sequence percentage before the merger of sequences
with pre.cluster and the removal of chimeras with VSEARCH. B) Chimera
sequence percentage before the removal of chimeras with VSEARCH. C) Full
pipeline. The error bars represent the 75\% interquartile range of the
median.

\textbf{Figure 6: The Correlation between Number of OTUs and Chimeras.}
A) Correlation before the merger of sequences with pre.cluster and the
removal of chimeras with VSEARCH. B) Correlation before the removal of
chimeras with VSEARCH. C) Correlation with full pipeline.

\newpage

\textbf{Figure S1: HiFi DNA Polymerase Nucleotide Subsitutions in Mock
Samples.}


\end{document}
