\documentclass[11pt,]{article}
\usepackage{lmodern}
\usepackage{amssymb,amsmath}
\usepackage{ifxetex,ifluatex}
\usepackage{fixltx2e} % provides \textsubscript
\ifnum 0\ifxetex 1\fi\ifluatex 1\fi=0 % if pdftex
  \usepackage[T1]{fontenc}
  \usepackage[utf8]{inputenc}
\else % if luatex or xelatex
  \ifxetex
    \usepackage{mathspec}
  \else
    \usepackage{fontspec}
  \fi
  \defaultfontfeatures{Ligatures=TeX,Scale=MatchLowercase}
\fi
% use upquote if available, for straight quotes in verbatim environments
\IfFileExists{upquote.sty}{\usepackage{upquote}}{}
% use microtype if available
\IfFileExists{microtype.sty}{%
\usepackage{microtype}
\UseMicrotypeSet[protrusion]{basicmath} % disable protrusion for tt fonts
}{}
\usepackage[margin=1.0in]{geometry}
\usepackage{hyperref}
\hypersetup{unicode=true,
            pdfborder={0 0 0},
            breaklinks=true}
\urlstyle{same}  % don't use monospace font for urls
\usepackage{graphicx,grffile}
\makeatletter
\def\maxwidth{\ifdim\Gin@nat@width>\linewidth\linewidth\else\Gin@nat@width\fi}
\def\maxheight{\ifdim\Gin@nat@height>\textheight\textheight\else\Gin@nat@height\fi}
\makeatother
% Scale images if necessary, so that they will not overflow the page
% margins by default, and it is still possible to overwrite the defaults
% using explicit options in \includegraphics[width, height, ...]{}
\setkeys{Gin}{width=\maxwidth,height=\maxheight,keepaspectratio}
\IfFileExists{parskip.sty}{%
\usepackage{parskip}
}{% else
\setlength{\parindent}{0pt}
\setlength{\parskip}{6pt plus 2pt minus 1pt}
}
\setlength{\emergencystretch}{3em}  % prevent overfull lines
\providecommand{\tightlist}{%
  \setlength{\itemsep}{0pt}\setlength{\parskip}{0pt}}
\setcounter{secnumdepth}{0}
% Redefines (sub)paragraphs to behave more like sections
\ifx\paragraph\undefined\else
\let\oldparagraph\paragraph
\renewcommand{\paragraph}[1]{\oldparagraph{#1}\mbox{}}
\fi
\ifx\subparagraph\undefined\else
\let\oldsubparagraph\subparagraph
\renewcommand{\subparagraph}[1]{\oldsubparagraph{#1}\mbox{}}
\fi

%%% Use protect on footnotes to avoid problems with footnotes in titles
\let\rmarkdownfootnote\footnote%
\def\footnote{\protect\rmarkdownfootnote}

%%% Change title format to be more compact
\usepackage{titling}

% Create subtitle command for use in maketitle
\newcommand{\subtitle}[1]{
  \posttitle{
    \begin{center}\large#1\end{center}
    }
}

\setlength{\droptitle}{-2em}
  \title{}
  \pretitle{\vspace{\droptitle}}
  \posttitle{}
  \author{}
  \preauthor{}\postauthor{}
  \date{}
  \predate{}\postdate{}

\usepackage{helvet} % Helvetica font
\renewcommand*\familydefault{\sfdefault} % Use the sans serif version of the font
\usepackage[T1]{fontenc}

\usepackage[none]{hyphenat}

\usepackage{setspace}
\doublespacing
\setlength{\parskip}{1em}

\usepackage{lineno}

\usepackage{pdfpages}

\usepackage{amsmath}

\usepackage{mathtools}

\begin{document}

\section{Assessing the Differences in 16S rRNA Gene Sequencing Due to
High Fidelity DNA
Polymerase}\label{assessing-the-differences-in-16s-rrna-gene-sequencing-due-to-high-fidelity-dna-polymerase}

\begin{center}
\vspace{25mm}

Marc A Sze${^1}$ and Patrick D Schloss${^1}$${^\dagger}$

\vspace{20mm}

$\dagger$ To whom correspondence should be addressed: pschloss@umich.edu

$1$ Department of Microbiology and Immunology, University of Michigan, Ann Arbor, MI




\end{center}

Co-author e-mails:

\begin{itemize}
\tightlist
\item
  \href{mailto:marcsze@med.umich.edu}{\nolinkurl{marcsze@med.umich.edu}}
\end{itemize}

\newpage

\linenumbers

\subsection{Abstract}\label{abstract}

\textbf{Background.} A typical 16S rRNA gene sequencing workflow can be
divided into preservation, extraction, amplification, and sequencing
steps. At each of these stages error can be introduced that will change
the underlying bacterial community composition results. In this study,
we focus on the amplification step's contribution to this overall error.
To accomplish this we assessed 16S rRNA gene sequencing results in human
fecal and mock community samples after using different high fidelity
(HiFi) DNA polymerases and number of amplification cycles.

\textbf{Methods.} We extracted DNA from fecal samples (n=4) using a
PowerMag DNA extraction kit with a 10 minute bead beating step and
amplified at 15, 20, 25, 30, and 35 cycles using Accuprime, Kappa,
Phusion, Platinum, or Q5 HiFi DNA polymerase. Amplification of mock
communities (technical replicates n=4) consisting of previously isolated
whole genomes of 8 different bacteria used the same approach. We first
assessed GC dependent differences, error rate, sequence error
prevalence, chimera prevalence, and correlation between chimera
prevalence and number of Operational Taxonomic Units (OTUs) by
polymerase and number of cycles. Next, differences in the number of OTUs
was examined based on the polymerase and number of cycles used.
Additionally, differences in the bacterial community composition by the
Bray-Curtis index also was assessed based on polymerase and number of
cycles. Based on these findings individual taxa differences based on
polymerase and number of cycles was investigated. Finally, Random Forest
models were created to test whether the bacterial community was better
at classifying polymerases, number of cycles, or individual donor. We
also assessed whether the most important taxa in the polymerase and
number of cycle Random Forest models were also important in the model
for individual donors.

\textbf{Results.} Predictably, we found noticeable differences in
relative abundance based on high and low GC content (P-value
\(\leqslant\) 0.04). Chimera prevalence in mock communities varied by
polymerase with differences being most notable at 35 cycles (Kappa =
5.71\% (median) versus Platinum = 26.62\%) and this variation persisted
after chimera removal using VSEARCH. We also observed positive
correlations between chimera prevalence and the number of OTUs with
Platinum having the highest (R\textsuperscript{2} = 0.974) and Kappa
having the worst (R\textsuperscript{2} = 0.478). When analyzing mock
community samples the variation in the number of OTUs detected by the
polymerases was observable as early as 20 cycles (P-value = 0.002).
There also was a large range in the number of OTUs amplified by the
polymerases at 35 cycles (Accuprime = 15 - 20 versus Phusion = 14 - 73).
When analyzing fecal samples we observed smaller differences in the
number of OTUs. Additionally, the median number of OTUs only varied by
HiFi DNA polymerase used at 35 cycles and not at 20 cycles like the mock
community (P-value \textless{} 0.0001). Random Forest models were most
successful at classifying individual donor samples rather than
polymerase or number of cycles used (P-value \(\leqslant\) 5.49e-07).
Additionally, the most important OTUs in the polymerase and number of
cycle models were not the most important in the individual donor sample
model.

\textbf{Conclusions.} Although there were 16S rRNA gene sequencing
differences based on polymerase and number of cycles used, bacterial
community composition differences are mostly only detectable in mock
communities. Collectively, these results provide evidence that smaller
biological differences between groups, based on 16S rRNA gene sequencing
of fecal samples, can be consistently detected regardless of polymerase
and number of cycles used.

\newpage

\subsection{Introduction}\label{introduction}

The bacterial community is reported to vary between case and control for
a number of diseases (Turnbaugh et al., 2008; Sze et al., 2015; Baxter
et al., 2016; Bonfili et al., 2017). However, for diseases like obesity,
the taxa identified have varied widely depending on the study (Turnbaugh
et al., 2008; Zupancic et al., 2012). Some of this variation could be
due to error introduced during the 16S rRNA gene sequencing workflow. A
typical 16S rRNA gene sequencing workflow can be divided into
preservation, extraction, PCR, and sequencing steps. The preservation
and extraction stages of the 16S rRNA gene sequencing workflow have been
the most extensively studied (Salter et al., 2014; Song et al., 2016;
Bassis et al., 2017; Kim et al., 2017). For preservation and extraction
stages of the workflow, it has been consistently found that there are
errors in the observed bacterial community composition based on the kits
used, but that these differences are smaller than the overall biological
difference measured between samples with different kits (Song et al.,
2016; Bassis et al., 2017). Since these studies use the same PCR
approach while varying preservation or extraction method, the
contribution of PCR error to this overall workflow is not as well
characterized.

There is a large body of literature that shows there are errors due to
primer and number of cycles chosen for the PCR stage of 16S rRNA gene
sequencing (Eckert \& Kunkel, 1991; Burkardt, 2000). Primers have
variable region dependent binding affinities which causes an inability
to detect specific bacteria (e.g.~V1-V3 does not detect
\emph{Haemophilus influenzae} and V3-V5 does not detect
\emph{Propionibacterium acnes}) (Sze et al., 2015 (Table S4); Meisel et
al., 2016). Another source of error is the selective amplification of
AT-rich over GC-rich sequences which exaggerate the difference between
16S rRNA genes higher in AT versus those higher in GC content (Polz \&
Cavanaugh, 1998). Many of these sources of error are made worse as the
number of cycles increases (Wang \& Wang, 1996; Haas et al., 2011;
Kebschull \& Zador, 2015). For example, both amplification error and
non-specific amplification (e.g.~incorrect amplicon size products) also
can increase as the number of cycles increases. This will increase the
number of Operational Taxonomic Units (OTUs) observed and drastically
change the values obtained from commonly used diversity measures (Acinas
et al., 2005; Santos et al., 2016). Additionally, as the number of
cycles increases more chimeras can form from an aborted extension step
that causes a priming error and subsequent secondary extension (Haas et
al., 2011). These chimeras will artificially increase community
diversity by increasing the number of OTUs that are observed (Haas et
al., 2011). In addition to these sources of errors, there also are
multiple families of DNA polymerases that have their own error rate and
proof reading capacity (Ishino \& Ishino, 2014). For simplicity, the use
of the term `error' within the context of this study will be used as a
broad term for changes in the downstream sequencing results due to the
amplification step (e.g.~higher GC content genomes, base substitutions,
chimeras, changes from the expected relative abundance, etc.).

Interestingly, the influence different DNA polymerases can have on the
observed 16S rRNA gene sequencing results have not been well studied
like some of the other sources of PCR-based error. A recent study found
differences in the number of OTUS and chimeras between normal and high
fidelity DNA polymerases (Gohl et al., 2016). The authors of this study
could reduce the difference between two polymerases by optimizing the
annealing and extension steps within the PCR protocol (Gohl et al.,
2016). Yet, within this study there was no comparison made between
different high fidelity DNA polymerases on the same samples. The authors
instead focused their efforts on testing specific types of amplification
methods. Additionally, the fecal samples that were analyzed, as a model
of a more complex community, may overestimate the biological variation
in human samples due to the comparison of captive versus semi-captive
red-shanked doucs. Due to these gaps, it makes sense to extend the
previous line of inquiry to include a more detailed examination of two
specific areas. First, whether differences in PCR error are detected
based on number of cycles used and high fidelity polymerases. Second,
whether these differences in error are enough to obscure smaller
biological signals than what has been previously reported (Gohl et al.,
2016).

This study will investigate whether there are error differences that are
dependent on high fidelity DNA polymerases and whether these differences
obscure the biological variation in the fecal bacterial communities
between individuals. We accomplish this by first examining GC-based
amplification error, sequence error (e.g.~substitutions), chimera
prevalence, the number of OTUs, Bray-Curtis index, and OTU-based
differences at varying number of cycles in five different high fidelity
DNA polymerases in mock communities. In addition to these metrics we
also created Random Forest models of the mock community to assess how
successful classification based on 16S rRNA gene sequencing of number of
cycles and type of polymerase could be. We then assessed if similar
differences in the number of OTUs, Bray-Curtis index, and OTUs could be
detected in human fecal samples. Additionally, we also created Random
Forest models to assess how successful they could be at classifying
fecal samples by number of cycles, polymerase used, or individual.
Collectively, our observations suggest that although detectable
differences occur based on number of cycles or high fidelity DNA
polymerase used, they are smaller than the biological differences
between individuals.

\newpage

\subsection{Results}\label{results}

\textbf{\emph{Error differences due to number of cycles or polymerase
used are detectable in mock communities.}} There was a significant
difference in relative abundance between high/low GC content based on
either the V4 16S rRNA gene region or the whole genome {[}Figure 1 \&
Table S1{]}. However, using only the GC content of the V4 region
resulted in less differences than when using the whole genome {[}Table
S1{]}. The differences between high and low GC content groups were
significant as early as 20 cycles in all polymerases except Q5 {[}Table
S1{]}. The highest (\emph{Staphylococcus}) and lowest
(\emph{Pseudomonas}) GC content bacteria were the most divergent from
the expected relative abundance of 0.12 {[}Figure 1{]}. These
observations suggest that the high fidelity DNA polymerases tested do
not prevent GC-based amplification error.

The median error rate varied by polymerase, with Kappa having the
highest error rate of all the polymerases across the number of cycles
{[}Figure 2A \& Table S2{]}. Thus, the majority of the differences
observed across the number of cycles was between Kappa and other
polymerases {[}Figure 2A \& Table S3{]}. The total sequences that
contained at least one error was also polymerase dependent, with the
majority of differences being between Kappa or Accuprime and the other
polymerases {[}Table S2 \& S3{]}. These differences in error rates were
not due to polymerase dependent differences in base substitution rate
{[}Figure S1{]}. Collectively, the results suggest that sequence error
is dependent on polymerase, persists across the number of cycles used,
and are not due to any bias towards a specific type of base
substitution.

We next examined whether chimeras also were dependent on polymerase and
whether this could affect the total number of OTUs. We observed
significant differences in the chimera prevalence based on polymerase
across the number of cycles used (P-value \textless{} 0.05) {[}Table
S2{]}. Differences in chimera prevalence between Platinum and all other
polymerases accounted for the majority of these differences {[}Table
S3{]}. Accuprime\textsuperscript{TM} had the lowest chimera prevalence
of all polymerases regardless of whether chimera removal with VSEARCH
was used {[}Figure 2B \& 2C{]}. The number of cycles used clearly
increased chimera prevalence for all polymerases but the rate of
increase differed {[}Figure 2B \& 2C{]}. There was also a plateau in the
total percent of chimeras that were removed that was similar for all
polymerases {[}Figure 2D{]}. Additionally, a positive correlation was
observed between chimeric sequences and the number of OTUs for all
polymerases {[}Figure 2E{]}. This positive correlation was strongest for
Accuprime\textsuperscript{TM}, Platinum, and Phusion {[}Figure 2E{]}.
This data suggests that chimera prevalence depends on polymerase, is
made worse by increasing the number of cycles, and confirms previous
reports that the number of OTUs is dependent on the prevalence of these
chimeric sequences.

\textbf{\emph{Bacterial community composition differences based on
number of cycles or polymerase used are less obvious in mock community
samples.}} Due to the close relationship between chimera prevalence and
the number of OTUs, we also observed a polymerase dependent difference
in the range of the number of OTUs in the mock community samples based
on number of cycles and polymerase {[}Figure 3A{]}. The closest any of
the polymerases came to a total of 8 OTUs, created by running the mock
reference 16S sequences through mothur processing, was at 25 and 30
cycles {[}Figure 3A{]}. Differences between polymerases for the number
of OTUs created were observed as early as 20 cycles in the mock
community (F-stat = 15.82, P-value = 0.002) {[}Table S4{]}. Using a
Tukey post-hoc test, the majority of differences for the number of OTUs
detected in the mock community was largely due to Kappa and Platinum
versus the other polymerases across the number of cycles used {[}Table
S5{]}.

We next investigated the effect polymerases and number of cycles had on
the bacterial community composition in more detail by using the
Bray-Curtis index. We observed clear differences in the mock community
based on polymerase used (P-value = 1e-04) as well as differences based
on polymerase and number of cycles (P-value = 1e-04). However, when
investigating specific cycles within each polymerase we observed no
difference in the bacterial community composition (P-value
\textgreater{} 0.05) {[}Figure 3B{]}. Finding specific differences in
overall bacterial community composition based on number of cycles and
polymerase, we next examined if specific taxa vary depending on
polymerase and number of cycles. We found that OTUs that taxonomically
classified to \emph{Salmonella}, \emph{Escherichia},
\emph{Enterococcus}, and \emph{Staphylococcus} differed based on
polymerase used {[}Table S6 \& S7{]}. These specific OTUs were all
members of the mock community and the differences were most pronounced
at 25 and 30 cycles {[}Table S7{]}. The majority of the differences
between taxa were due to differences between a single polymerase (Kappa)
and the other polymerases tested {[}Table S7{]}.

Based on these observations, we created Random Forest models to classify
samples based on number of cycles or polymerase used. When assessing the
Random Forest models there was a difference between models built to
classify number of cycles versus those built to classify polymerase used
{[}P-value = 5.06e-07{]}. Although there was a difference, neither model
was successful at correctly classifying samples based on 16S rRNA gene
sequencing data (cycles model, probability of correct classification =
0.42 (0.4 - 0.43)(min - max) versus polymerase model, probability of
correct classification = 0.39 (0.37 - 0.4)). Our observations in mock
communities suggest that although differences are detected based on
number of cycles or polymerase used, the overall community-wide
differences are subtle.

\textbf{\emph{Few differences based on number of cycles and polymerase
used were identified in fecal samples.}} Unlike mock community samples,
an overall difference in the number of OTUs was only detected between
polymerases at 35 cycles in fecal samples (F-stat \textgreater{} 16.35,
P-value = 9.7e-05) {[}Table S4{]}, and a Tukey post-hoc test failed to
identify any differences between specific polymerases used (P-value
\textgreater{} 0.05) {[}Table S5{]}. However, differences in the range
of the number of OTUs detected for fecal samples was dependent on the
polymerase used (e.g.~Accuprime at 35 cycles = 84 - 106 versus Phusion
at 35 cycles = 84 - 136) {[}Figure 4A{]}. There was also a trend for
lower number of cycles (15-20) to result in a reduced range in the
number of OTUs versus higher number of cycles (25, 30, and 35) for all
polymerases (e.g.~Phusion at 15 cycles = 10 - 19 versus Phusion at 35
cycles = 84 - 136) {[}Figure 4A{]}. Based on these observations, there
are only small differences in the total number of OTUs detected in fecal
samples that are based on number of cycles or polymerases used.

When using the Bray-Curtis index, we observed that polymerases and
number of cycles had no affect on the bacterial community composition in
fecal samples (P-value = 1). There was also no interaction between
individuals, number of cycles, and polymerase used (P-value = 1). When
using PERMANOVA to test for community differences based on any of the
number of cycles within polymerases, only Phusion had cycle dependent
differences (P-value = 0.03). Additionally, Phusion was one of two
polymerases that had enough sequences to be rarefied to 1000 sequences
at 15 cycles. Within each respective 5-cycle increment comparison there
was no difference in Bray-Curtis index between the polymerases (P-value
\textgreater{} 0.05) {[}Figure 4B{]}. However, there was an overall
decrease in Bray-Curtis index by cycle comparison group (e.g.~the 15
cycle versus 20 cycle group compared to the 30 cycle versus 35 cycle
group) (P-value \textless{} 0.01) {[}Figure 4B{]}. Using a Dunn's
post-hoc test the 15 cycle versus 20 cycle and 20 cycle versus 25 cycle
comparison groups had a higher Bray-Curtis index than the 30 cycle
versus 35 cycle comparison group (P-value \textless{} 0.05). In addition
to assessing community composition by Bray-Curtis index, we also
investigated differences in OTUs based on polymerase at each number of
cycle used and no significant differences could be found (P-value
\(\geqslant\) 0.01, corrected P-value = 1). Overall, these results
suggest that in fecal samples the number of cycles and polymerase used
only minimally change the bacterial community composition.

Finally, we built Random Forest models to classify samples for the
number of cycles, polymerase used, or individual it was obtained from.
Using 16S rRNA gene sequencing data there was a large disparity in the
model's ability to correctly classify samples based on polymerase used,
number of cycles, and individual (P-value \(\leqslant\) 5.49e-07). There
was a clear difference in the success of the model used to classify
individuals (probability of correct classification = 0.87 (0.86 - 0.9))
versus the models created to classify number of cycles or polymerase
(cycles model, probability of correct classification = 0.26 (0.25 -
0.27)(min - max) and polymerase model, probability of correct
classification = 0.18 (0.17 - 0.2)). Additionally, the top 10 most
important OTUs to the polymerase and number of cycle models were not the
most important OTUs to the model that classified individuals {[}Figure
5{]}. This suggests that OTUs that change the most due to the number of
cycles and polymerase in fecal samples are not the same as OTUs that are
part of the biological variation between individuals {[}Figure 5{]}.
Collectively, our observations show that despite a variety of errors
associated with the amplification process, it is still possible to
detect the bacterial community variation between individuals.

\newpage

\subsection{Discussion}\label{discussion}

Previous studies have reported GC dependent amplification differences
(Polz \& Cavanaugh, 1998) and we confirm that these differences still
persist in high fidelity DNA polymerase when analyzing mock community
samples {[}Figure 1{]}. Additionally, these specific polymerases have
different sequence error rates and chimera prevalence {[}Figure 2{]}.
Chimera prevalence in particular was strongly associated with the total
number of OTUs detected in our mock community and vaired by polymerase
{[}Figure 2E \& 3A{]}. Additionally, the bacterial community composition
varied by number of cycles and polymerase in these mock community
samples. Clearly, it is possible to identify a large number of
differences based on number of cycles and polymerase used when analyzing
mock community samples. However, these differences may not be a major
concern when analyzing more complex community samples.

Fecal samples showed a similar increase in median and range for the
number of OTUs, based on number of cycles and polymerase used, as that
found when analyzing mock community samples {[}Figure 4A{]}. However,
there were few differences between the polymerases within each of the
amplification cycles {[}Figure 4A{]}. Additionally, these differences
did not translate to large bacterial community composition changes based
on number of cycles or polymerase. The differences that were identifed
were small and occured based on the number of cycles, independent of the
polymerase used {[}Figure 4B{]}. There was also no OTU-based differences
observed within the number of cycles between any of the polymerases for
fecal samples. Interestingly, Random Forest models built on this OTU
data had the highest classification success for individuals versus
number of cycles or polymerase. In addition to the better model
performance of Random Forest for individuals, the most important OTUs to
this model were not the most important OTUs for the other models tested
{[}Figure 5{]}. Collectively, these observations show that differences
based on the number of cycles and between high fidelity DNA polymerases
can be found. However, when using 16S rRNA gene sequencing these
technical differences can be smaller than the biological differences
between fecal bacterial communities within individuals.

The small differences observed due to error rate and chimera prevalence
may be due to the actual DNA polymerase family being used within the PCR
mix. DNA polymerases from distinct families have different binding
affinities and error correction capacity (Ishino \& Ishino, 2014).
However, based on the observations within our study this is unlikely to
be the only contributor. Within our study the highest and lowest chimera
prevalence both belonged to a family A polymerase (Platinum and
Accuprime\textsuperscript{TM} respectively) (Ishino \& Ishino, 2014).
Additionally, based on the information supplied by the respective
manufacturers, the differences between the two PCR mixtures are not
immediately apparent. Both PCR mixtures contain a recombinant \emph{Taq}
DNA polymerase, a \emph{Pyrococcus} spp GB-D polymerase and a platinum
\emph{Taq} antibody. Since it is not possible to know everything about
the mixture beyond what was willingly provided by the manufacturer, it
is possible that differences in how the recombinant \emph{Taq} was
generated or other compounds within the PCR mixture could be a
contributing factor for the differences in error rate and chimera
prevalence. Beyond the choice of the type of polymerase, there may be
other ways to reduce the affect of polymerase dependent error rates and
chimera prevalence on the downstream results.

Based on polymerase used, our observations generally found no difference
when using the Bray-Curtis index to measure the bacterial community.
Specifically, there was no difference in distance between successive
5-cycle increments within samples between the polymerases when using the
Bray-Curtis index {[}Figure 4{]}. One possible reason for this outcome
is that many of the OTUs generated by polymerase dependent error rates
and chimera prevalence are likely not highly abundant, allowing the
Bray-Curtis index to be able to successfully down-weight these OTUs
(Minchin, 1987). The choice of downstream diversity metric could be an
important consideration in helping to mitigate some of the observed
polymerase dependent differences in error rate and chimera prevalence.
Metrics that solely use presence/absence of OTUs (e.g.~Jaccard (Real \&
Vargas, 1996), richness) may be less robust to polymerase dependent
error rates and chimera prevalence. When choosing a distance metric,
careful consideration of the biases introduced from the PCR step of the
16S rRNA gene sequencing workflow need to be taken into account. One
possible way to better choose polymerases may be based on the DNA
polymerase family used in the PCR mixture.

Collectively, our data shows that using distinct high fidelity DNA
polymerases will result in a different number of OTUs being detected.
The differences in the number of OTUs are primarily due to chimera
prevalence and sequence error rates that are distinct to the specific
polymerase. Despite this variation in the number of OTUs, our Random
Forest models were able to correctly classify individuals regardless of
the number of cycles and polymerase used. These results suggest that
even though there is a large variation in the high fidelity DNA
polymerases used across studies, biological variation similar in effect
to fecal bacterial community composition differences between individuals
will still be consistent.

\newpage

\subsection{Conclusion}\label{conclusion}

Although care should always be taken when choosing a polymerase for 16S
rRNA gene sequencing, our observations show that the differences between
a variety of polymerases are dwarfed by the actual biological variation
in fecal communities between individuals. Therefore, if the biological
signal of interest is similar to differences in fecal bacterial
communities found between individuals, then the type of high fidelity
DNA polymerase used will only minimally change the results.

\newpage

\subsection{Materials \& Methods}\label{materials-methods}

\textbf{\emph{Human and mock samples.}} Fecal samples were obtained from
4 individuals who were part of the Enterics Research Investigational
Network (ERIN). The processing and storage of these samples were
previously published (Seekatz et al., 2016). Other than confirmation
that none of these individuals had a \emph{Clostridium difficle}
infection, clinical data and other types of meta data were not utilized
or accessed for this study. All samples were extracted using the
MOBIO\textsuperscript{TM} PowerMag Microbiome RNA/DNA extraction kit
(now Qiagen, MD, USA). The ZymoBIOMICS\textsuperscript{TM} Microbial
Community DNA Standard (Zymo, CA, USA) was used for mock communities and
was made up of \emph{Pseudomonas aeruginosa}, \emph{Escherichia coli},
\emph{Salmonella enterica}, \emph{Lactobacillus fermentum},
\emph{Enterococcus faecalis}, \emph{Staphylococcus aureus},
\emph{Listeria monocytogenes}, and \emph{Bacillus subtilis} at equal
genomic DNA abundance
(\url{http://www.zymoresearch.com/microbiomics/microbial-standards/zymobiomics-microbial-community-standards}).

\textbf{\emph{PCR protocol.}} The five different high fidelity DNA
polymerases (hereto referred to as polymerases) that were tested
included AccuPrime\textsuperscript{TM} (ThermoFisher, MA, USA), KAPA
HIFI (Roche, IN, USA), Phusion (ThermoFisher, MA, USA), Platinum
(ThermoFisher, MA, USA), and Q5 (New England Biolabs, MA, USA). The
polymerases activation time was 2 minutes, unless a different activation
was specified by the manufacturer. The annealing and extension time for
Platinum and Accuprime followed a previously published protocol (Kozich
et al., 2013)
(\url{https://github.com/SchlossLab/MiSeq_WetLab_SOP/blob/master/MiSeq_WetLab_SOP_v4.md}).
For Kappa and Q5, the annealing and extension time also were from a
previously published protocol (Gohl et al., 2016). For Phusion, the
company defined activation and annealing times were used while the
extension time followed the Accuprime and Platinum settings.

The number of cycles in the PCR for fecal and mock samples started at 15
and increased by 5 up to 35 cycles, with amplicons used at each 5-step
increase for sequencing. The PCR of fecal DNA samples consisted of all 4
samples at 15, 20, 25, 30, and 35 cycles for each polymerase (total
sample n=100). The mock communities had 4 replicates at 15, 20, 25, and
35 cycles and 10 replicates for 30 cycles for all polymerases (total
samples n=130). No mock community sample had enough PCR product at 15
cycles for adequate 16S rRNA gene sequencing.

\textbf{\emph{Sequence processing.}} The mothur software program was
used for all sequence processing steps (Schloss et al., 2009). The
protocol has been previously published (Kozich et al., 2013)
(\url{https://www.mothur.org/wiki/MiSeq_SOP}). Two major differences
from the published protocol were the use of VSEARCH instead of UCHIME
for chimera detection and the use of the OptiClust algorithm instead of
average neighbor for OTU generation at 97\% similarity (Edgar et al.,
2011; Rognes et al., 2016; Westcott \& Schloss, 2017). Sequence error
was determined using the `seq.error' command on mock samples to compare
back to the reference 16S sequences of \emph{P. aeruginosa}, \emph{E.
coli}, \emph{S. enterica}, \emph{L. fermentum}, \emph{E. faecalis},
\emph{S. aureus}, \emph{L. monocytogenes}, and \emph{B. subtilis}
(Schloss et al., 2009; Cole et al., 2013; Rognes et al., 2016). For
simplicity, the use of the term `error' within the context of this study
moving forward encompasses changes in the downstream sequencing results
due to the amplification step (e.g.~higher GC content genomes, base
substitutions, deletions, insertions, non-specific amplification,
chimeras, etc.).

\textbf{\emph{Analysis workflow.}} To adjust for unequal sequencing, all
samples were rarefied to 1000 sequences for downstream analysis. The
analysis of the mock community of each polymerase for GC-based
amplification differences, sequence error rate, number of sequences with
an error, base substitution, and numbers of chimeras before and after
chimera removal with VSEARCH was assessed. Additionally, the correlation
between the number of chimeras and the number of OTUs was also assessed.
The total number of OTUs, taxa differences, and Bray-Curtis indices were
analyzed for both the fecal and mock community samples. Finally, Random
Forest models were created to assess whether classification of
polymerase, number of cycles, or individual could be performed best
using 16S rRNA gene sequencing data. For the model used to classify
individuals, all samples were included from all number of cycles and
polymerases but only the individual labels were used. Both number of
cycles and polymerase models included all samples but only the number of
cycles or polymerase label was used for each respective model.
Additionally, overlap between the most important OTUs to the three
models was assessed using mean decrease in accuracy (MDA).

\textbf{\emph{Statistical analysis.}} All analysis was done with the R
(v 3.4.4) software package (R Core Team, 2017). Data transformation and
graphing was completed using the tidyverse package (v 1.2.1) and colors
selected using the viridis package (v 0.4.1) (Garnier, 2017; Wickham,
2017). High and low GC content was determined based on the median GC
percentage of either the V4 region or the whole genome of the bacterial
species used in the mock community. Differences in the total number of
OTUs were analyzed using an ANOVA with a tukey post-hoc test. For the
comparison of the number of OTUS in fecal samples the data was
normalized to each individual by cycle number to account for the
biological variation. Bray-Curtis distance matrices were generated using
mothur. The distance matrix data was analyzed using PERMANOVA with the
vegan package (v 2.4.5) (Oksanen et al., 2017) and Kruskal-Wallis tests
within R. The Random Forest models were run using the caret package (v
6.0.78) (Jed Wing et al., 2017). A total of 100 10-fold CV runs on
different 80/20 splits of the data was run to generate a range of the
Logloss value. The probability of a correct call was obtained from this
Logloss value by taking the negative natural logarithm. For both error
and chimera analysis, samples were tested using Kruskal-Wallis with a
Dunns post-hoc test. Where applicable, correction for multiple
comparison utilized the Benjamini-Hochberg method (Benjamini \&
Hochberg, 1995).

\textbf{\emph{Reproducible methods.}} The code and analysis can be found
here \url{https://github.com/SchlossLab/Sze_PCRSeqEffects_XXXX_2017}.
The raw sequences can be found on the SRA at the following accession
number SRP132931.

\newpage

\subsection{Acknowledgements}\label{acknowledgements}

The authors would like to thank all the study participants in ERIN whose
samples were utilized. We also would like to thank Judy Opp and April
Cockburn for their effort in sequencing the samples as part of the
Microbiome Core Facility at the University of Michigan. Additional
thanks to members of the Schloss lab and Dr.~Marcy Balunas for reading
earlier drafts of the manuscript and providing helpful critiques. Salary
support for Marc A. Sze came from the Canadian Institute of Health
Research and NIH grant UL1TR002240. Salary support for Patrick D.
Schloss came from NIH grants P30DK034933 and 1R01CA215574.

\newpage

\subsection{References}\label{references}

\hypertarget{refs}{}
\hypertarget{ref-Acinas2005}{}
Acinas SG., Sarma-Rupavtarm R., Klepac-Ceraj V., Polz MF. 2005.
PCR-induced sequence artifacts and bias: Insights from comparison of two
16S rRNA clone libraries constructed from the same sample. \emph{Applied
and Environmental Microbiology} 71:8966--8969. DOI:
\href{https://doi.org/10.1128/aem.71.12.8966-8969.2005}{10.1128/aem.71.12.8966-8969.2005}.

\hypertarget{ref-storage_Bassis_2017}{}
Bassis CM., Nicholas M. Moore., Lolans K., Seekatz AM., Weinstein RA.,
Young VB., Hayden MK. 2017. Comparison of stool versus rectal swab
samples and storage conditions on bacterial community profiles.
\emph{BMC Microbiology} 17. DOI:
\href{https://doi.org/10.1186/s12866-017-0983-9}{10.1186/s12866-017-0983-9}.

\hypertarget{ref-Baxter2016}{}
Baxter NT., Ruffin MT., Rogers MAM., Schloss PD. 2016. Microbiota-based
model improves the sensitivity of fecal immunochemical test for
detecting colonic lesions. \emph{Genome Medicine} 8. DOI:
\href{https://doi.org/10.1186/s13073-016-0290-3}{10.1186/s13073-016-0290-3}.

\hypertarget{ref-benjamini_controlling_1995}{}
Benjamini Y., Hochberg Y. 1995. Controlling the false discovery rate: A
practical and powerful approach to multiple testing. \emph{Journal of
the Royal Statistical Society. Series B (Methodological)} 57:289--300.

\hypertarget{ref-Bonfili2017}{}
Bonfili L., Cecarini V., Berardi S., Scarpona S., Suchodolski JS.,
Nasuti C., Fiorini D., Boarelli MC., Rossi G., Eleuteri AM. 2017.
Microbiota modulation counteracts alzheimer's disease progression
influencing neuronal proteolysis and gut hormones plasma levels.
\emph{Scientific Reports} 7. DOI:
\href{https://doi.org/10.1038/s41598-017-02587-2}{10.1038/s41598-017-02587-2}.

\hypertarget{ref-Burkardt2000}{}
Burkardt H-J. 2000. Standardization and quality control of PCR analyses.
\emph{Clinical Chemistry and Laboratory Medicine} 38. DOI:
\href{https://doi.org/10.1515/cclm.2000.014}{10.1515/cclm.2000.014}.

\hypertarget{ref-rdp_Cole_2013}{}
Cole JR., Wang Q., Fish JA., Chai B., McGarrell DM., Sun Y., Brown CT.,
Porras-Alfaro A., Kuske CR., Tiedje JM. 2013. Ribosomal database
project: Data and tools for high throughput rRNA analysis. \emph{Nucleic
Acids Research} 42:D633--D642. DOI:
\href{https://doi.org/10.1093/nar/gkt1244}{10.1093/nar/gkt1244}.

\hypertarget{ref-Eckert1991}{}
Eckert KA., Kunkel TA. 1991. DNA polymerase fidelity and the polymerase
chain reaction. \emph{Genome Research} 1:17--24. DOI:
\href{https://doi.org/10.1101/gr.1.1.17}{10.1101/gr.1.1.17}.

\hypertarget{ref-uchime_Edgar_2011}{}
Edgar RC., Haas BJ., Clemente JC., Quince C., Knight R. 2011. UCHIME
improves sensitivity and speed of chimera detection.
\emph{Bioinformatics} 27:2194--2200. DOI:
\href{https://doi.org/10.1093/bioinformatics/btr381}{10.1093/bioinformatics/btr381}.

\hypertarget{ref-viridis_citation_2017}{}
Garnier S. 2017. \emph{Viridis: Default color maps from 'matplotlib'}.

\hypertarget{ref-taq_Gohl_2016}{}
Gohl DM., Vangay P., Garbe J., MacLean A., Hauge A., Becker A., Gould
TJ., Clayton JB., Johnson TJ., Hunter R., Knights D., Beckman KB. 2016.
Systematic improvement of amplicon marker gene methods for increased
accuracy in microbiome studies. \emph{Nature Biotechnology} 34:942--949.
DOI: \href{https://doi.org/10.1038/nbt.3601}{10.1038/nbt.3601}.

\hypertarget{ref-Haas2011}{}
Haas BJ., Gevers D., Earl AM., Feldgarden M., Ward DV., Giannoukos G.,
Ciulla D., Tabbaa D., Highlander SK., Sodergren E., Methe B., DeSantis
TZ., Petrosino JF., Knight R., and BWB. 2011. Chimeric 16S rRNA sequence
formation and detection in sanger and 454-pyrosequenced PCR amplicons.
\emph{Genome Research} 21:494--504. DOI:
\href{https://doi.org/10.1101/gr.112730.110}{10.1101/gr.112730.110}.

\hypertarget{ref-polymerase_Ishino_2014}{}
Ishino S., Ishino Y. 2014. DNA polymerases as useful reagents for
biotechnology â the history of developmental research in the field.
\emph{Frontiers in Microbiology} 5. DOI:
\href{https://doi.org/10.3389/fmicb.2014.00465}{10.3389/fmicb.2014.00465}.

\hypertarget{ref-caret_citation}{}
Jed Wing MKC from., Weston S., Williams A., Keefer C., Engelhardt A.,
Cooper T., Mayer Z., Kenkel B., R Core Team., Benesty M., Lescarbeau R.,
Ziem A., Scrucca L., Tang Y., Candan C., Hunt. T. 2017. \emph{Caret:
Classification and regression training}.

\hypertarget{ref-Kebschull2015}{}
Kebschull JM., Zador AM. 2015. Sources of PCR-induced distortions in
high-throughput sequencing data sets. \emph{Nucleic Acids
Research}:gkv717. DOI:
\href{https://doi.org/10.1093/nar/gkv717}{10.1093/nar/gkv717}.

\hypertarget{ref-review_Kim_2017}{}
Kim D., Hofstaedter CE., Zhao C., Mattei L., Tanes C., Clarke E., Lauder
A., Sherrill-Mix S., Chehoud C., Kelsen J., Conrad M., Collman RG.,
Baldassano R., Bushman FD., Bittinger K. 2017. Optimizing methods and
dodging pitfalls in microbiome research. \emph{Microbiome} 5. DOI:
\href{https://doi.org/10.1186/s40168-017-0267-5}{10.1186/s40168-017-0267-5}.

\hypertarget{ref-protocol_Kozich_2013}{}
Kozich JJ., Westcott SL., Baxter NT., Highlander SK., Schloss PD. 2013.
Development of a dual-index sequencing strategy and curation pipeline
for analyzing amplicon sequence data on the MiSeq illumina sequencing
platform. \emph{Applied and Environmental Microbiology} 79:5112--5120.
DOI: \href{https://doi.org/10.1128/aem.01043-13}{10.1128/aem.01043-13}.

\hypertarget{ref-Meisel2016}{}
Meisel JS., Hannigan GD., Tyldsley AS., SanMiguel AJ., Hodkinson BP.,
Zheng Q., Grice EA. 2016. Skin microbiome surveys are strongly
influenced by experimental design. \emph{Journal of Investigative
Dermatology} 136:947--956. DOI:
\href{https://doi.org/10.1016/j.jid.2016.01.016}{10.1016/j.jid.2016.01.016}.

\hypertarget{ref-bc_index_Minchin1987}{}
Minchin PR. 1987. An evaluation of the relative robustness of techniques
for ecological ordination. \emph{Vegetatio} 69:89--107. DOI:
\href{https://doi.org/10.1007/bf00038690}{10.1007/bf00038690}.

\hypertarget{ref-vegan_citation}{}
Oksanen J., Blanchet FG., Friendly M., Kindt R., Legendre P., McGlinn
D., Minchin PR., O'Hara RB., Simpson GL., Solymos P., Stevens MHH.,
Szoecs E., Wagner H. 2017. \emph{Vegan: Community ecology package}.

\hypertarget{ref-polz_bias_1998}{}
Polz MF., Cavanaugh CM. 1998. Bias in template-to-product ratios in
multitemplate PCR. \emph{Applied and Environmental Microbiology}
64:3724--3730.

\hypertarget{ref-r_citation_2017}{}
R Core Team. 2017. \emph{R: A language and environment for statistical
computing}. Vienna, Austria: R Foundation for Statistical Computing.

\hypertarget{ref-Real1996}{}
Real R., Vargas JM. 1996. The probabilistic basis of jaccards index of
similarity. \emph{Systematic Biology} 45:380--385. DOI:
\href{https://doi.org/10.1093/sysbio/45.3.380}{10.1093/sysbio/45.3.380}.

\hypertarget{ref-vsearch_Rognes_2016}{}
Rognes T., Flouri T., Nichols B., Quince C., Mahé F. 2016. VSEARCH: A
versatile open source tool for metagenomics. \emph{PeerJ} 4:e2584. DOI:
\href{https://doi.org/10.7717/peerj.2584}{10.7717/peerj.2584}.

\hypertarget{ref-contamination_Salter2014}{}
Salter SJ., Cox MJ., Turek EM., Calus ST., Cookson WO., Moffatt MF.,
Turner P., Parkhill J., Loman NJ., Walker AW. 2014. Reagent and
laboratory contamination can critically impact sequence-based microbiome
analyses. \emph{BMC Biology} 12. DOI:
\href{https://doi.org/10.1186/s12915-014-0087-z}{10.1186/s12915-014-0087-z}.

\hypertarget{ref-BautistadelosSantos2016}{}
Santos QMB-d los., Schroeder JL., Blakemore O., Moses J., Haffey M.,
Sloan W., Pinto AJ. 2016. The impact of sampling, PCR, and sequencing
replication on discerning changes in drinking water bacterial community
over diurnal time-scales. \emph{Water Research} 90:216--224. DOI:
\href{https://doi.org/10.1016/j.watres.2015.12.010}{10.1016/j.watres.2015.12.010}.

\hypertarget{ref-mothur_schloss_2009}{}
Schloss PD., Westcott SL., Ryabin T., Hall JR., Hartmann M., Hollister
EB., Lesniewski RA., Oakley BB., Parks DH., Robinson CJ., Sahl JW.,
Stres B., Thallinger GG., Horn DJV., Weber CF. 2009. Introducing mothur:
Open-source, platform-independent, community-supported software for
describing and comparing microbial communities. \emph{Applied and
Environmental Microbiology} 75:7537--7541. DOI:
\href{https://doi.org/10.1128/aem.01541-09}{10.1128/aem.01541-09}.

\hypertarget{ref-erin_seekatz_2016}{}
Seekatz AM., Rao K., Santhosh K., Young VB. 2016. Dynamics of the fecal
microbiome in patients with recurrent and nonrecurrent clostridium
difficile infection. \emph{Genome Medicine} 8. DOI:
\href{https://doi.org/10.1186/s13073-016-0298-8}{10.1186/s13073-016-0298-8}.

\hypertarget{ref-preservation_Song_2016}{}
Song SJ., Amir A., Metcalf JL., Amato KR., Xu ZZ., Humphrey G., Knight
R. 2016. Preservation methods differ in fecal microbiome stability,
affecting suitability for field studies. \emph{mSystems} 1:e00021--16.
DOI:
\href{https://doi.org/10.1128/msystems.00021-16}{10.1128/msystems.00021-16}.

\hypertarget{ref-Sze2015}{}
Sze MA., Dimitriu PA., Suzuki M., McDonough JE., Campbell JD., Brothers
JF., Erb-Downward JR., Huffnagle GB., Hayashi S., Elliott WM., Cooper
J., Sin DD., Lenburg ME., Spira A., Mohn WW., Hogg JC. 2015. Host
response to the lung microbiome in chronic obstructive pulmonary
disease. \emph{American Journal of Respiratory and Critical Care
Medicine} 192:438--445. DOI:
\href{https://doi.org/10.1164/rccm.201502-0223oc}{10.1164/rccm.201502-0223oc}.

\hypertarget{ref-Turnbaugh2008}{}
Turnbaugh PJ., Hamady M., Yatsunenko T., Cantarel BL., Duncan A., Ley
RE., Sogin ML., Jones WJ., Roe BA., Affourtit JP., Egholm M., Henrissat
B., Heath AC., Knight R., Gordon JI. 2008. A core gut microbiome in
obese and lean twins. \emph{Nature} 457:480--484. DOI:
\href{https://doi.org/10.1038/nature07540}{10.1038/nature07540}.

\hypertarget{ref-Wang1996}{}
Wang GCY., Wang Y. 1996. The frequency of chimeric molecules as a
consequence of PCR co-amplification of 16S rRNA genes from different
bacterial species. \emph{Microbiology} 142:1107--1114. DOI:
\href{https://doi.org/10.1099/13500872-142-5-1107}{10.1099/13500872-142-5-1107}.

\hypertarget{ref-opticlust_Westcott_2017}{}
Westcott SL., Schloss PD. 2017. OptiClust, an improved method for
assigning amplicon-based sequence data to operational taxonomic units.
\emph{mSphere} 2:e00073--17. DOI:
\href{https://doi.org/10.1128/mspheredirect.00073-17}{10.1128/mspheredirect.00073-17}.

\hypertarget{ref-tidyverse_2017}{}
Wickham H. 2017. \emph{Tidyverse: Easily install and load 'tidyverse'
packages}.

\hypertarget{ref-Zupancic2012}{}
Zupancic ML., Cantarel BL., Liu Z., Drabek EF., Ryan KA., Cirimotich S.,
Jones C., Knight R., Walters WA., Knights D., Mongodin EF., Horenstein
RB., Mitchell BD., Steinle N., Snitker S., Shuldiner AR., Fraser CM.
2012. Analysis of the gut microbiota in the old order amish and its
relation to the metabolic syndrome. \emph{PLoS ONE} 7:e43052. DOI:
\href{https://doi.org/10.1371/journal.pone.0043052}{10.1371/journal.pone.0043052}.

\newpage

\textbf{Figure 1: Relative abundance differences due to GC is consistent
across polymerases.} At each number of cycle the points and lines
represent the median and with the minimum and maximum relative abundance
respectively. Regardless of the number of cycle used a consistent
difference between high and low GC bacteria was observed. The dotted
line represents the actual relative abundance that each bacterium should
be at within the mock community sample.

\textbf{Figure 2: Error rate and chimera prevalence vary by polymerase
and affect the number of observed OTUs.} A) The error bars represent the
75\% interquartile range of the median error rate. B) Percentage of
chimeric sequences without the removal of chimeras with VSEARCH. C)
Percentage of chimeric sequences with the removal of chimeras with
VSEARCH. D) The total percent of chimeric sequences removed with VSEARCH
by cycle number. The error bars represent the 75\% interquartile range
of the median. E) Chimera prevalence and the the observed number of OTUs
are strongly correlated in mock community samples.

\textbf{Figure 3: Subtle differences based on number of cycles and
polymerase used are detected in bacterial community composition of mock
samples.} A) The range in the number of OTUs detected in the mock
samples increased as cycle number increased. This range was larger for
specific HiFi DNA polymerases. The points represent the median number of
OTUs and the lines represent the range from the minimum to maximum
number of OTUs detected within the four technical replicates. The dotted
black line represents the number of OTUs detected when only the
references sequences for the mock community are clustered. A) Within
replicate difference based on the next 5-cycle PCR interval in mock
samples. The lines represent the range of the minimum and maximum
Bray-Curtis index value for each PCR 5-cycle increment comparison. The
closer a sample is to a Bray-Curtis index of 1.00 the more dissimilar
the bacterial community is of the two compared number of cycles.

\textbf{Figure 4: Subtle differences based on number of cycles and
polymerase used are detected in bacterial community composition of fecal
samples.} A) The range in the number of OTUs detected in the different
fecal samples increased as cycle number increased. This range was larger
for specific HiFi DNA polymerases. The points represent the median
number of OTUs and the lines represent the range from the minimum to
maximum number of OTUs detected within the four fecal samples. B) Within
person differences based on the next 5-cycle PCR interval in fecal
samples. The points represent the median Bray-Curtis index for the
samples. The lines represent the range of the minimum and maximum
Bray-Curtis index value for each PCR 5-cycle increment comparison. The
closer a sample is to a Bray-Curtis index of 1.00 the more dissimilar
the bacterial community is of the two compared number of cycles.

\textbf{Figure 5: Important OTUs for the number of cycles and polymerase
models are not the most important OTUs in the model classifying
individuals.} Color highlights the top 10 OTUs in number of cycles and
polymerase used and where they fall in the model used to classify
individuals. The majority of these top 10 OTUs have 0 importance to the
model used to classify individuals.

\newpage

\textbf{Figure S1: No preference for specific base subsitutions was
observed across the polymerases in mock community samples.}


\end{document}
