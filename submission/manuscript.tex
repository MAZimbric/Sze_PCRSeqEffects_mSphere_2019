\documentclass[12pt,]{article}
\usepackage{lmodern}
\usepackage{amssymb,amsmath}
\usepackage{ifxetex,ifluatex}
\usepackage{fixltx2e} % provides \textsubscript
\ifnum 0\ifxetex 1\fi\ifluatex 1\fi=0 % if pdftex
  \usepackage[T1]{fontenc}
  \usepackage[utf8]{inputenc}
\else % if luatex or xelatex
  \ifxetex
    \usepackage{mathspec}
  \else
    \usepackage{fontspec}
  \fi
  \defaultfontfeatures{Ligatures=TeX,Scale=MatchLowercase}
\fi
% use upquote if available, for straight quotes in verbatim environments
\IfFileExists{upquote.sty}{\usepackage{upquote}}{}
% use microtype if available
\IfFileExists{microtype.sty}{%
\usepackage{microtype}
\UseMicrotypeSet[protrusion]{basicmath} % disable protrusion for tt fonts
}{}
\usepackage[margin=1.0in]{geometry}
\usepackage{hyperref}
\hypersetup{unicode=true,
            pdfborder={0 0 0},
            breaklinks=true}
\urlstyle{same}  % don't use monospace font for urls
\usepackage{graphicx,grffile}
\makeatletter
\def\maxwidth{\ifdim\Gin@nat@width>\linewidth\linewidth\else\Gin@nat@width\fi}
\def\maxheight{\ifdim\Gin@nat@height>\textheight\textheight\else\Gin@nat@height\fi}
\makeatother
% Scale images if necessary, so that they will not overflow the page
% margins by default, and it is still possible to overwrite the defaults
% using explicit options in \includegraphics[width, height, ...]{}
\setkeys{Gin}{width=\maxwidth,height=\maxheight,keepaspectratio}
\IfFileExists{parskip.sty}{%
\usepackage{parskip}
}{% else
\setlength{\parindent}{0pt}
\setlength{\parskip}{6pt plus 2pt minus 1pt}
}
\setlength{\emergencystretch}{3em}  % prevent overfull lines
\providecommand{\tightlist}{%
  \setlength{\itemsep}{0pt}\setlength{\parskip}{0pt}}
\setcounter{secnumdepth}{0}
% Redefines (sub)paragraphs to behave more like sections
\ifx\paragraph\undefined\else
\let\oldparagraph\paragraph
\renewcommand{\paragraph}[1]{\oldparagraph{#1}\mbox{}}
\fi
\ifx\subparagraph\undefined\else
\let\oldsubparagraph\subparagraph
\renewcommand{\subparagraph}[1]{\oldsubparagraph{#1}\mbox{}}
\fi

%%% Use protect on footnotes to avoid problems with footnotes in titles
\let\rmarkdownfootnote\footnote%
\def\footnote{\protect\rmarkdownfootnote}

%%% Change title format to be more compact
\usepackage{titling}

% Create subtitle command for use in maketitle
\newcommand{\subtitle}[1]{
  \posttitle{
    \begin{center}\large#1\end{center}
    }
}

\setlength{\droptitle}{-2em}
  \title{}
  \pretitle{\vspace{\droptitle}}
  \posttitle{}
  \author{}
  \preauthor{}\postauthor{}
  \date{}
  \predate{}\postdate{}

\usepackage{helvet} % Helvetica font
\renewcommand*\familydefault{\sfdefault} % Use the sans serif version of the font
\usepackage[T1]{fontenc}

\usepackage[none]{hyphenat}

\usepackage{setspace}
\doublespacing
\setlength{\parskip}{1em}

\usepackage{lineno}

\usepackage{pdfpages}

\usepackage{amsmath}

\usepackage{mathtools}

\begin{document}

\section{What the Taq? The Influence of Taq Polymerase on 16S rRNA
Sequencing}\label{what-the-taq-the-influence-of-taq-polymerase-on-16s-rrna-sequencing}

\begin{center}
\vspace{25mm}

Marc A Sze${^1}$ and Patrick D Schloss${^1}$${^\dagger}$

\vspace{20mm}

$\dagger$ To whom correspondence should be addressed: pschloss@umich.edu

$1$ Department of Microbiology and Immunology, University of Michigan, Ann Arbor, MI




\end{center}

Co-author e-mails:

\begin{itemize}
\tightlist
\item
  \href{mailto:marcsze@med.umich.edu}{\nolinkurl{marcsze@med.umich.edu}}
\end{itemize}

\newpage

\linenumbers

\subsection{Abstract}\label{abstract}

\textbf{Background.}

\textbf{Methods.}

\textbf{Results.}

\textbf{Conclusions.}

\newpage

\subsection{Introduction}\label{introduction}

\newpage

\subsection{Materials \& Methods}\label{materials-methods}

\textbf{\emph{Insert Sub section here}}

\textbf{\emph{Reproducible Methods:}} The code and analysis can be found
here \url{https://github.com/SchlossLab/Sze_PCRSeqEffects_XXXX_2017}.

\newpage

\subsection{Results}\label{results}

\textbf{\emph{Insert Sub title here}}

\newpage

\subsection{Discussion}\label{discussion}

\newpage

\subsection{Conclusion}\label{conclusion}

\newpage

\subsection{Acknowledgements}\label{acknowledgements}

The authors would like to thank all the study participants who were a
part of each of the individual studies utilized. We would also like to
thank each of the study authors for making their data available for use.
Finally we would like to thank the members of the Schloss lab for
valuable feed back and proof reading during the formulation of this
manuscript.

\newpage

\subsection{References}\label{references}

\hypertarget{refs}{}

\newpage

\textbf{Table 1: }

\footnotesize

\normalsize
\newpage

\textbf{Table 2: }

\footnotesize

\normalsize
\newpage

\textbf{Figure 1: .}

\textbf{Figure 2: .}

\textbf{Figure 3: .}

\textbf{Figure 4: .}

\textbf{Figure 5: .}

\textbf{Figure 6: .}

\textbf{Figure 7: .}

\newpage

\textbf{Figure S1: .}

\textbf{Figure S2: .}

\textbf{Figure S3: .}

\textbf{Figure S4: .}

\textbf{Figure S5: .}

\textbf{Figure S6: .}

\textbf{Figure S7: .}

\newpage


\end{document}
